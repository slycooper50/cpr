% From mitthesis package
% Version: 1.04, 2023/10/19
% Documentation: https://ctan.org/pkg/mitthesis
\chapter{Introduction}
The use of models is part of our everyday life. In our modern society, we are sourounded by a myriad of complex systems that are
essential for the workings of our lifes from transportation, communication and food processing to energy production. Scientis and engineers
handle complexity by building simplified models that can be further enhanced to increase their ability to describe the complex systems.
In the petroleum industry, reservoir simulation is a branch of reservoir engineering that aims to develop models of the petroleum systems.
These models are built using knowledge of geological, geophysical and petrophysical properties of the petroleum reservoir system.
They are routinely used to describe fluid flow during the phase of field development planning and production, which enable the engineers 
to develop an optimum field production plan. The importance of reservoir simulation stems from the complexity of the physical systems that reservoir 
engineers attempt to describe. The use of mathematical models is essential to provide precise and valuable answers to the questions the simulator is 
set to answer. 

\begin{figure}[htb]
\centering
\resizebox{15cm}{!}{\pgfdeclarelayer{behind}
\pgfdeclarelayer{background}
\pgfdeclarelayer{foreground}
\pgfsetlayers{behind,background,main,foreground}
\tikzset{box/.style={draw, rectangle, rounded corners, thick, node 
 distance=7em, 
text width=6em, text centered, minimum height=3.5em}}
\tikzset{every node/.append style={font=\scriptsize}}

%************************************************************
%************************************************************
%  Define block styles
%************************************************************
%************************************************************
\tikzset{block/.style={rectangle split, draw, rectangle split parts=2,
text width=14em, text centered, rounded corners, minimum height=4em},
grnblock/.style={rectangle, draw, fill=green!20, text width=10em, text 
centered, rounded corners, minimum height=4em}, 
whtblock/.style={rectangle, draw, fill=white!20, text width=10em, text 
centered, minimum height=3em},    
ylwblock/.style={rectangle, draw, fill=yellow, text width=10em, text 
centered, minimum height=3em}, 
line/.style={draw, -{Latex[length=2mm,width=1mm]}},
cloud/.style={draw, ellipse,fill=white!20, node distance=3cm,minimum height=3em},  
container1/.style={draw, rectangle,dashed,inner sep=0.28cm, rounded
corners,fill=green!8,minimum height=2.4cm,minimum width=5.6cm},
container2/.style={draw, rectangle,dashed,inner sep=0.28cm, rounded
corners,fill=red!8,minimum height=4.cm,minimum width=6.3cm},
container3/.style={draw, rectangle,dashed,inner sep=0.28cm, rounded
corners,fill=gray!15,minimum height=3cm,minimum width=6.3cm},
arrow/.style={-, thick}
}
%************************************************************
%************************************************************ 
\begin{tikzpicture}[node distance = 1.20cm, auto]
%************************************************************
%************************************************************
%  Draw nodes
%************************************************************
\node [whtblock,font=\fontsize{10}{0}\selectfont] (resgrid) {Reservoir geometrical representation: grid data points.};
%===============================================    
%  Constituent elastic parameters
%===============================================  
\node [block, below=of resgrid,rectangle split part fill= 
{orange!20,blue!3},font=\fontsize{10}{0}\selectfont, yshift=1cm] (petro) 
{\textbf{Petrophysical Parameters}
\nodepart[text width=3cm]{two} 	$\phi,\ \boldmath{\tensor{K}},\ \Delta x,\ \Delta y,\ \Delta z$\\
				$P_{c}(S_{w}),\ K_{rw},\ K_{ro},\ K_{rg}$};
% ****************************************************
% ****************************************************
 \node [right=3.9cm of resgrid, yshift=1.2cm] [fill=white,draw,font=\fontsize{12}{0}\selectfont] 
 (flowmodel) {\textbf{Flow Model}};

 \node [below=2.7cm of resgrid, xshift=4.5cm] [fill=white,draw,font=\fontsize{12}{0}\selectfont] 
 (wellmodel) {\textbf{Well Model}};

\node [below=of wellmodel, yshift=0cm,block,anchor=center,rectangle split part fill= {orange!20,blue!3},font=\fontsize{10}{0}\selectfont] (welleq) 
{\textbf{Well Model Equations}
\nodepart[text width=5cm] 
{two}$q_{\alpha,i} = J_{i}\lambda_{\alpha,i}(p_{i} - p_{bhp} - H_{wi})$\\
$J = \frac{\theta K h}{\ln{r_{e}/r_{w}}+S}$\\
[1em] 
}; 

\node [below=of flowmodel, yshift=-0.5cm,block,anchor=center,rectangle split part fill= {orange!20,blue!3},font=\fontsize{10}{0}\selectfont] (MatParm) 
{\textbf{Mathematical Model Of Flow (PDE)}
\nodepart[text width=5cm] 
{two}$\partial_{t}(\phi b_{w}S_{w}) + \nabla \cdot (b_{w}\vec{u_{w}}) = b_{w}q_{w}$\\
$\partial_{t}(\phi b_{o}S_{o}) + \nabla \cdot (b_{o}\vec{u_{o}}) = b_{o}q_{o}$\\
$\partial_{t}\phi[b_{g}S_{g}+b_{o}R_{s}S_{o}] + \nabla \cdot (b_{g}\vec{u_{g}} + b_{o}R_{s}\vec{u_{o}}) = b_{g}q_{g} + b_{o}R_{s}q_{o}$\\
[1em] 
}; 

\begin{pgfonlayer}{background}
 \coordinate (aux1) at ([yshift=3mm]resgrid.north);
 \node [container1,fit=(aux1)(petro)] (cont3) 
 {};
   \node at (cont3.north) [fill=white,draw,font=\fontsize{12}{0}\selectfont] 
 {\textbf{Geological Properties}};
 %-----------------------------------------------------------
\end{pgfonlayer}

\begin{pgfonlayer}{behind}
 \coordinate (aux2) at ([yshift=-8mm, xshift=5.3cm]resgrid.east);
 \node[container2, fit=(aux2)] (cont1) 
 {};
\end{pgfonlayer}

\begin{pgfonlayer}{behind}
 \coordinate (aux2) at ([yshift=-15mm, xshift=-3cm]cont1.south);
 \node[container3, fit=(aux2), yshift=-0.8cm] (cont2) 
 {};
%\draw[color=red] (cont1.south) to (cont2.east);
\draw[arrow](cont1.south)+(1, 0) -- ++(1,-1) |- (cont2.east);
\draw[arrow](cont3.south)+(-1, 0) -- ++(-1,-1) |- (cont2.west);
\draw[arrow](cont3.east)+(0,0.04) -- (cont1.west);
\end{pgfonlayer}
\end{tikzpicture}
}
\caption{Models used in a reservoir simulator.}\label{models}
\end{figure}

There are several physical systems within a petroleum reservoir that a reservoir simulator needs to model. 
These can include the grid dimensions representing the reservoir and the geological properties of the reservoir such as porosity, 
permeability and multiphase fluid flow parameters like capillary pressure and relative permeability. A fluid flow model that consists of 
conservation laws and a constitutive Darcy's law for flow in porous media must also be incorporated as a model to the reservoir simulator.
Several fluid modeling techniques are used in the industry, the most commonly known are the \textit{Black-oil} and \textit{Compositional} models.
Moreover, the boundary conditions for the differential equations describing the fluid flow are the wells, both injectors and producers. 
The wells are also routinely modeled in the simulator using differential equations to describe inflow performance of the wells. One of well models
main objectives is to relate the reservoir gridblock pressure to the well bottom hole pressure. More sophisticated well models handle cross flow and
multilateral, horizontal and slanted wells. Reservoir simulators can also include additional models to simulate networks of fractures, surface facilities,
enhanced oil recovery and geomechanics. The main standard models that a simple reservoir simulator must incorporate are summarized in figure \ref{models}.

The inner components of reservoir simulators are more or less universal. Each simulator has to support a 
gridding infrastructure to support discretization of the simulation domain, a nonlinear solver (usually based on Newton's Method) to 
solve the nonlinear coupled PDEs of fluid flow, a linear solver that will solve the matrix produced by the nonlinear solver, a timestepping
mechanism guided by some convergence criteria of the nonlinear solver and an I/O system to generate results requested by the user for post-processing.

\begin{figure}[htb]
\raggedright
\resizebox{15cm}{!}{\begin{tikzpicture} [
output/.style    = { rectangle, draw=black, thick, 
    fill=embeddings_color,
    rounded corners, minimum height=2em },
post/.style    = { rectangle, draw=black, thick, 
    fill=emb_color, text width=4em, text centered,
    rounded corners, minimum height=2em },
addnorm/.style    = { rectangle, draw=black, thick, 
    fill=cadmgreen!40, text width=10em, text centered,
    rounded corners, minimum height=2em },
four/.style    = { rectangle, draw=black, thick, 
    fill=fourier_color, text width=10em, text centered,
    rounded corners, minimum height=4.5em },
grid/.style    = { rectangle, draw=black, thick, 
    fill=frenchblue, text width=10em, text centered,
    rounded corners, minimum height=2em },
conn/.style    = { rectangle, draw=black, thick, 
    fill=frenchblue, text width=10em, text centered,
    rounded corners, minimum height=2em },
main/.style    = { trapezium, draw=black, thick, 
    fill=boston!70, text width=10em, text centered,
    rounded corners, minimum height=2em },
end/.style    = { trapezium , trapezium angle=-67.5, draw=black, thick, 
    fill=boston!70, text width=10em, text centered,
    rounded corners, minimum height=2em },
input/.style    = { rectangle, draw=black, thick, 
    fill=frenchblue, text width=10em, text centered,
    rounded corners, minimum height=2em },
conv/.style    = { diamond, draw=black, thick, 
    fill=frenchblue, text centered,
    rounded corners},
line/.style     = { draw, thick, <-, shorten >=0pt },
]
\definecolor{emb_color}{RGB}{252,224,225}
\definecolor{embeddings_color}{RGB}{232,204,205}
\definecolor{cadmgreen}{rgb}{0.0, 0.42, 0.24}
\definecolor{chocolate}{RGB}{210,105,30}
\definecolor{alizarin}{rgb}{0.82, 0.1, 0.26}
\definecolor{airforceblue}{rgb}{0.36, 0.54, 0.66}
\definecolor{boston}{rgb}{0.8, 0.0, 0.0}
\definecolor{frenchblue}{rgb}{0.0, 0.45, 0.73}
\definecolor{fourier_color}{RGB}{252,226,187}
% Define nodes in a matrix
\matrix [column sep=1mm, row sep=5mm] {
    & \node [main] (main) {\textbf{Simulator Main}};         & \\
    & \node [input] (input) {\textit{Input Parsing}};  & \\
    & \node [grid] (grid) {\textit{Grid Construction}};                     & \\
    & \node [conn] (connect) {\textit{Connectivity Graph}};                     & \\\\
    & \node [addnorm] (res) {\texttt{Compute Residual}};       & \\
    & \node [addnorm] (jac) {\texttt{Compute Jacobian}};       & \\
    & \coordinate (null2);                                & \\
	& \node [four] (cpr) {\texttt{Precondition}};        & \\
	& \node [four] (krylov) {\texttt{Krylov Iterator}};                & \\
    & \node [conv] (conv1) {converged};                & \\
    & \coordinate (null1);                                & \\
};

\node[below = 1cm of conv1] (output) {Output Of A Timestep};
\node[conv, below = 1cm of output, line width=0.03cm,draw, rounded corners=0.3cm,] (conv2) {\shortstack{End Of \\\\Simulation \\ \\Time}};
\node [yshift=0.5cm]at (conv2.west) {No};
\node [xshift=1cm, yshift=-1.7cm]at (conv2.west) {Yes};

\begin{pgfonlayer}{background}
    \node[fit=(output),output](post){};
    \coordinate (N1) at (post.west|-krylov);
    \coordinate (N2) at (post.east|-krylov);
    \node[inner xsep=120pt,inner ysep=14pt,fit=(res)(krylov)(N1)(N2)(conv1),fill=airforceblue!40, line width=0.03cm,draw, rounded corners=0.3cm,label={[align=left, xshift=-8.5cm,yshift=-3cm]\textbf{\textit{Time Loop}}}](timeloop){};
    \node[inner xsep=55pt,inner ysep=8pt,fit=(cpr)(krylov),fill=alizarin!40, line width=0.03cm,draw, rounded corners=0.3cm,label={[align=left,xshift=-5.2cm,yshift=1cm]\textbf{\textit{Newton's}} \\\textit{\textbf{Loop}}}](newton){};
    \node[inner xsep=2pt,inner ysep=4pt,fit=(cpr)(krylov),fill=chocolate!50, line width=0.03cm,draw, rounded corners=0.3cm,label={[align=left,xshift=-3.2cm,yshift=-2cm]\textit{\textbf{Solver's}} \\\textit{\textbf{Loop}}}](solver){};
    \node [yshift=0.5cm]at (conv1.east) {No};
    \node [xshift=0.5cm, yshift=-1cm]at (conv1.west) {Yes};
\end{pgfonlayer}

\node[end, below= 5mm of conv2](end){\textbf{End}};

% connect all nodes defined above
\begin{scope} [every path/.style=line]
\path (input)       --  (main);
\path (grid)       --  (input);
\path (connect)       --  (grid);
\path (timeloop)        --  (connect);
\path (jac)          --  (res);
\path (solver)          --  (jac);
\path (solver.north)+(0,0.2)       --++  (3,0.2) |- (solver.east);
\path (krylov)     --  (cpr);
\path (post)  --  (timeloop) coordinate[midway](null1) ;
\path (timeloop.north)+(0,0.3)  --++  (-7,0.3) -- (-7,-12.65) -| (conv2.west);
\path  (conv1.north) -- (solver.south);
\path  (res.east)--++(3,0) |-(conv1.east);
\path (conv2) -- (post);
\path (end) -- (conv2);
\end{scope}
\path [draw] (timeloop.south)--(conv1);

\end{tikzpicture}
}
\caption{Generic Algorithm Of A Reservoir Simulator.}\label{simulator}
\end{figure}


A crucial advantage of simulation tools over classical engineering analysis practices is the ability
to perform a large number of calculations in an efficient way. In reservoir simulation, the most taxing
component of runing a simulator is usually the linear solver. Therefore, research in developing efficient
techniques in solving large system of equations is ample in the reservoir simulation literature. The main
focus is usually on developing scalable preconditioners that can be combined with iterative Krylov-based 
iterators. The industry standard preconditioner is the Constrained Pressure Residual. This preconditioner
has its roots in the works of Wallis\supercite{Wallis_1983,Wallis_1985}, the essential idea relies on decoupling
the resulting system of linear equations by isolating the elliptic dominant part of the system (pressure) from 
the hyperbolic (saturation or concentration) part of the system. 

The convergance of a Krylov iterative solver depends essentially on the preconditioner. The prototypical Krylov method, the
Conjugate Gradient can handle Symmetric-Positive-Definite matrices. For reservoir simulation cases, where matrices are ill-conditioned and
not symmetric, the Generalized Minimum Residual \textit{GMRES}, is used\supercite{roy}. 

\section{Spatial Discretization}


\section{Temporal Discretization}

\section{Nonlinear Equations Solution Method}

\section{Linear System Solution Method}
