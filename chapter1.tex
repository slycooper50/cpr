% From mitthesis package
% Version: 1.04, 2023/10/19
% Documentation: https://ctan.org/pkg/mitthesis
\chapter{Introduction}
The inner components of reservoir simulators are more or less universal. Each simulator has to support a 
gridding infrastructure to support discretization of the simulation domain, a nonlinear solver (usually based on Newton's Method) to 
solve the nonlinear coupled PDEs of fluid flow, a linear solver that will solve the matrix produced by the nonlinear solver, a timestepping
mechanism guided by some convergence criteria of the nonlinear solver and an I/O system to generate results requested by the user for post-processing.

A crucial advantage of simulation tools over classical engineering analysis practices is the ability
to perform a large number of calculations in an efficient way. In reservoir simulation, the most taxing
component of runing a simulator is usually the linear solver. Therefore, research in developing efficient
techniques in solving large system of equations is ample in the reservoir simulation literature. The main
focus is usually on developing scalable preconditioners that can be combined with iterative Krylov-based 
iterators. The industry standard preconditioner is the Constrained Pressure Residual. This preconditioner
has its roots in the works of Wallis\supercite{Wallis_1983,Wallis_1985}, the essential idea relies on decoupling
the resulting system of linear equations by isolating the elliptic dominant part of the system (pressure) from 
the hyperbolic (saturation or concentration) part of the system. 

The convergance of a Krylov iterative solver depends essentially on the preconditioner. The prototypical Krylov method, the
Conjugate Gradient can handle Symmetric-Positive-Definite matrices. For reservoir simulation cases, where matrices are ill-conditioned and
not symmetric, the Generalized Minimum Residual \textit{GMRES}, is used\supercite{roy}. 



