% From mitthesis package
% Version: 1.04, 2023/10/19
% Documentation: https://ctan.org/pkg/mitthesis
\chapter{Introduction}
The use of models is part of our everyday life. In our modern society, we are surrounded by a myriad of complex systems that are
essential for the workings of our lives from transportation, communication and food processing to energy production. Scientists and engineers
handle complexity by building simplified models that can be further enhanced to increase their ability to describe the complex systems.
In the petroleum industry, reservoir simulation is a branch of reservoir engineering that aims to develop models of the petroleum systems.
These models are built using knowledge of geological, geophysical and petrophysical properties of the petroleum reservoir system.
They are routinely used to describe fluid flow during the phase of field development planning and production, which enables the engineers 
to develop an optimum field production plan. The importance of reservoir simulation stems from the complexity of the physical systems that reservoir 
engineers attempt to describe. The use of mathematical models is essential to provide precise and valuable answers to the questions the simulator is 
set to answer. 

\begin{figure}[htb]
\centering
\resizebox{15cm}{!}{\pgfdeclarelayer{behind}
\pgfdeclarelayer{background}
\pgfdeclarelayer{foreground}
\pgfsetlayers{behind,background,main,foreground}
\tikzset{box/.style={draw, rectangle, rounded corners, thick, node 
 distance=7em, 
text width=6em, text centered, minimum height=3.5em}}
\tikzset{every node/.append style={font=\scriptsize}}

%************************************************************
%************************************************************
%  Define block styles
%************************************************************
%************************************************************
\tikzset{block/.style={rectangle split, draw, rectangle split parts=2,
text width=14em, text centered, rounded corners, minimum height=4em},
grnblock/.style={rectangle, draw, fill=green!20, text width=10em, text 
centered, rounded corners, minimum height=4em}, 
whtblock/.style={rectangle, draw, fill=white!20, text width=10em, text 
centered, minimum height=3em},    
ylwblock/.style={rectangle, draw, fill=yellow, text width=10em, text 
centered, minimum height=3em}, 
line/.style={draw, -{Latex[length=2mm,width=1mm]}},
cloud/.style={draw, ellipse,fill=white!20, node distance=3cm,minimum height=3em},  
container1/.style={draw, rectangle,dashed,inner sep=0.28cm, rounded
corners,fill=green!8,minimum height=2.4cm,minimum width=5.6cm},
container2/.style={draw, rectangle,dashed,inner sep=0.28cm, rounded
corners,fill=red!8,minimum height=4.cm,minimum width=6.3cm},
container3/.style={draw, rectangle,dashed,inner sep=0.28cm, rounded
corners,fill=gray!15,minimum height=3cm,minimum width=6.3cm},
arrow/.style={-, thick}
}
%************************************************************
%************************************************************ 
\begin{tikzpicture}[node distance = 1.20cm, auto]
%************************************************************
%************************************************************
%  Draw nodes
%************************************************************
\node [whtblock,font=\fontsize{10}{0}\selectfont] (resgrid) {Reservoir geometrical representation: grid data points.};
%===============================================    
%  Constituent elastic parameters
%===============================================  
\node [block, below=of resgrid,rectangle split part fill= 
{orange!20,blue!3},font=\fontsize{10}{0}\selectfont, yshift=1cm] (petro) 
{\textbf{Petrophysical Parameters}
\nodepart[text width=3cm]{two} 	$\phi,\ \boldmath{\tensor{K}},\ \Delta x,\ \Delta y,\ \Delta z$\\
				$P_{c}(S_{w}),\ K_{rw},\ K_{ro},\ K_{rg}$};
% ****************************************************
% ****************************************************
 \node [right=3.9cm of resgrid, yshift=1.2cm] [fill=white,draw,font=\fontsize{12}{0}\selectfont] 
 (flowmodel) {\textbf{Flow Model}};

 \node [below=2.7cm of resgrid, xshift=4.5cm] [fill=white,draw,font=\fontsize{12}{0}\selectfont] 
 (wellmodel) {\textbf{Well Model}};

\node [below=of wellmodel, yshift=0cm,block,anchor=center,rectangle split part fill= {orange!20,blue!3},font=\fontsize{10}{0}\selectfont] (welleq) 
{\textbf{Well Model Equations}
\nodepart[text width=5cm] 
{two}$q_{\alpha,i} = J_{i}\lambda_{\alpha,i}(p_{i} - p_{bhp} - H_{wi})$\\
$J = \frac{\theta K h}{\ln{r_{e}/r_{w}}+S}$\\
[1em] 
}; 

\node [below=of flowmodel, yshift=-0.5cm,block,anchor=center,rectangle split part fill= {orange!20,blue!3},font=\fontsize{10}{0}\selectfont] (MatParm) 
{\textbf{Mathematical Model Of Flow (PDE)}
\nodepart[text width=5cm] 
{two}$\partial_{t}(\phi b_{w}S_{w}) + \nabla \cdot (b_{w}\vec{u_{w}}) = b_{w}q_{w}$\\
$\partial_{t}(\phi b_{o}S_{o}) + \nabla \cdot (b_{o}\vec{u_{o}}) = b_{o}q_{o}$\\
$\partial_{t}\phi[b_{g}S_{g}+b_{o}R_{s}S_{o}] + \nabla \cdot (b_{g}\vec{u_{g}} + b_{o}R_{s}\vec{u_{o}}) = b_{g}q_{g} + b_{o}R_{s}q_{o}$\\
[1em] 
}; 

\begin{pgfonlayer}{background}
 \coordinate (aux1) at ([yshift=3mm]resgrid.north);
 \node [container1,fit=(aux1)(petro)] (cont3) 
 {};
   \node at (cont3.north) [fill=white,draw,font=\fontsize{12}{0}\selectfont] 
 {\textbf{Geological Properties}};
 %-----------------------------------------------------------
\end{pgfonlayer}

\begin{pgfonlayer}{behind}
 \coordinate (aux2) at ([yshift=-8mm, xshift=5.3cm]resgrid.east);
 \node[container2, fit=(aux2)] (cont1) 
 {};
\end{pgfonlayer}

\begin{pgfonlayer}{behind}
 \coordinate (aux2) at ([yshift=-15mm, xshift=-3cm]cont1.south);
 \node[container3, fit=(aux2), yshift=-0.8cm] (cont2) 
 {};
%\draw[color=red] (cont1.south) to (cont2.east);
\draw[arrow](cont1.south)+(1, 0) -- ++(1,-1) |- (cont2.east);
\draw[arrow](cont3.south)+(-1, 0) -- ++(-1,-1) |- (cont2.west);
\draw[arrow](cont3.east)+(0,0.04) -- (cont1.west);
\end{pgfonlayer}
\end{tikzpicture}
}
\caption{Models used in a reservoir simulator.}\label{models}
\end{figure}

There are several physical systems within a petroleum reservoir that a reservoir simulator needs to model. 
These can include the grid dimensions representing the reservoir and the geological properties of the reservoir such as porosity, 
permeability and multiphase fluid flow parameters like capillary pressure and relative permeability. A fluid flow model that consists of 
conservation laws and a constitutive Darcy's law for flow in porous media must also be incorporated as a model to the reservoir simulator.
Several fluid modeling techniques are used in the industry, the most commonly known are the \textit{Black-oil} and \textit{Compositional} models.
Moreover, the boundary conditions for the differential equations describing the fluid flow are the wells, both injectors and producers. 
The wells are also routinely modeled in the simulator using algebraic equations to describe inflow performance of the wells. One of the well models
main objectives is to relate the reservoir gridblock pressure to the well bottom hole pressure. More sophisticated well models handle cross flow and
multilateral, horizontal and slanted wells. Reservoir simulators can also include additional models to simulate networks of fractures, surface facilities,
enhanced oil recovery and geomechanics. The main standard models that a simple Black-Oil reservoir simulator must incorporate are summarized in Figure \ref{models}.

The inner components of reservoir simulators are more or less universal. Each simulator has to support a 
gridding infrastructure to support discretization of the simulation domain, a nonlinear solver (usually based on Newton's Method) to 
solve the nonlinear coupled PDEs of fluid flow, a linear solver that will solve the matrix produced by the nonlinear solver, a timestepping
mechanism guided by some convergence criteria of the nonlinear solver and an I/O system to generate results requested by the user for post-processing.

At the \textit{Center for Subsurface Energy and the Environment}, there are three variants of reservoir simulators. The full-fledged compositional 
simulator \texttt{UTCOMPRS} is capable of handling up to four-phase flow, an aqueous, oleic, gaseous and an additional nonaqueous liquid phase. 
The simulator was originally developed in the 1990s\supercite{utcomp}. The second simulator \texttt{UTCHEMRS}, is specialized in chemical flooding.
\texttt{UTCOMPRS} and \texttt{UTCHEMRS} supports an Adaptive-Implicit formulation, with gridding supported by both Cartesian and the industry standard
Corner-Point Geometry. The third \texttt{Multi-Purpose Simulator (MPS)} is focused on performing parallel simulation for large complex reservoirs operating under
various recovery processes. The modeling group maintains the development of all three simulators. 

The overall algorithm of a reservoir simulator can be simplified by focusing on the essential components. These will be found in every reservoir simulator, whether
its focused on research or industrial activities. The simulation process is depicted in Figure \ref{simulator}. The process commences by parsing the input files that describe the
reservoir and boundary conditions as specified by wells or aquifers. Then an algorithm to build the simulation grid is run and connectivity graph is constructed. Then the three main
loops of the simulator are entered. These loops consist of a time loop that will finish once the end of specified simulation time is reached. The second inner loop is Newtonian loop,
which will end once the set of nonlinear equations are solved with acceptable tolerance. The third inner loop is the linear solver loop that will finish once the linear system of equations
is solved with acceptable tolerance.

\begin{figure}[htb]
\raggedright
\resizebox{15cm}{!}{\begin{tikzpicture} [
output/.style    = { rectangle, draw=black, thick, 
    fill=embeddings_color,
    rounded corners, minimum height=2em },
post/.style    = { rectangle, draw=black, thick, 
    fill=emb_color, text width=4em, text centered,
    rounded corners, minimum height=2em },
addnorm/.style    = { rectangle, draw=black, thick, 
    fill=cadmgreen!40, text width=10em, text centered,
    rounded corners, minimum height=2em },
four/.style    = { rectangle, draw=black, thick, 
    fill=fourier_color, text width=10em, text centered,
    rounded corners, minimum height=4.5em },
grid/.style    = { rectangle, draw=black, thick, 
    fill=frenchblue, text width=10em, text centered,
    rounded corners, minimum height=2em },
conn/.style    = { rectangle, draw=black, thick, 
    fill=frenchblue, text width=10em, text centered,
    rounded corners, minimum height=2em },
main/.style    = { trapezium, draw=black, thick, 
    fill=boston!70, text width=10em, text centered,
    rounded corners, minimum height=2em },
end/.style    = { trapezium , trapezium angle=-67.5, draw=black, thick, 
    fill=boston!70, text width=10em, text centered,
    rounded corners, minimum height=2em },
input/.style    = { rectangle, draw=black, thick, 
    fill=frenchblue, text width=10em, text centered,
    rounded corners, minimum height=2em },
conv/.style    = { diamond, draw=black, thick, 
    fill=frenchblue, text centered,
    rounded corners},
line/.style     = { draw, thick, <-, shorten >=0pt },
]
\definecolor{emb_color}{RGB}{252,224,225}
\definecolor{embeddings_color}{RGB}{232,204,205}
\definecolor{cadmgreen}{rgb}{0.0, 0.42, 0.24}
\definecolor{chocolate}{RGB}{210,105,30}
\definecolor{alizarin}{rgb}{0.82, 0.1, 0.26}
\definecolor{airforceblue}{rgb}{0.36, 0.54, 0.66}
\definecolor{boston}{rgb}{0.8, 0.0, 0.0}
\definecolor{frenchblue}{rgb}{0.0, 0.45, 0.73}
\definecolor{fourier_color}{RGB}{252,226,187}
% Define nodes in a matrix
\matrix [column sep=1mm, row sep=5mm] {
    & \node [main] (main) {\textbf{Simulator Main}};         & \\
    & \node [input] (input) {\textit{Input Parsing}};  & \\
    & \node [grid] (grid) {\textit{Grid Construction}};                     & \\
    & \node [conn] (connect) {\textit{Connectivity Graph}};                     & \\\\
    & \node [addnorm] (res) {\texttt{Compute Residual}};       & \\
    & \node [addnorm] (jac) {\texttt{Compute Jacobian}};       & \\
    & \coordinate (null2);                                & \\
	& \node [four] (cpr) {\texttt{Precondition}};        & \\
	& \node [four] (krylov) {\texttt{Krylov Iterator}};                & \\
    & \node [conv] (conv1) {converged};                & \\
    & \coordinate (null1);                                & \\
};

\node[below = 1cm of conv1] (output) {Output Of A Timestep};
\node[conv, below = 1cm of output, line width=0.03cm,draw, rounded corners=0.3cm,] (conv2) {\shortstack{End Of \\\\Simulation \\ \\Time}};
\node [yshift=0.5cm]at (conv2.west) {No};
\node [xshift=1cm, yshift=-1.7cm]at (conv2.west) {Yes};

\begin{pgfonlayer}{background}
    \node[fit=(output),output](post){};
    \coordinate (N1) at (post.west|-krylov);
    \coordinate (N2) at (post.east|-krylov);
    \node[inner xsep=120pt,inner ysep=14pt,fit=(res)(krylov)(N1)(N2)(conv1),fill=airforceblue!40, line width=0.03cm,draw, rounded corners=0.3cm,label={[align=left, xshift=-8.5cm,yshift=-3cm]\textbf{\textit{Time Loop}}}](timeloop){};
    \node[inner xsep=55pt,inner ysep=8pt,fit=(cpr)(krylov),fill=alizarin!40, line width=0.03cm,draw, rounded corners=0.3cm,label={[align=left,xshift=-5.2cm,yshift=1cm]\textbf{\textit{Newton's}} \\\textit{\textbf{Loop}}}](newton){};
    \node[inner xsep=2pt,inner ysep=4pt,fit=(cpr)(krylov),fill=chocolate!50, line width=0.03cm,draw, rounded corners=0.3cm,label={[align=left,xshift=-3.2cm,yshift=-2cm]\textit{\textbf{Solver's}} \\\textit{\textbf{Loop}}}](solver){};
    \node [yshift=0.5cm]at (conv1.east) {No};
    \node [xshift=0.5cm, yshift=-1cm]at (conv1.west) {Yes};
\end{pgfonlayer}

\node[end, below= 5mm of conv2](end){\textbf{End}};

% connect all nodes defined above
\begin{scope} [every path/.style=line]
\path (input)       --  (main);
\path (grid)       --  (input);
\path (connect)       --  (grid);
\path (timeloop)        --  (connect);
\path (jac)          --  (res);
\path (solver)          --  (jac);
\path (solver.north)+(0,0.2)       --++  (3,0.2) |- (solver.east);
\path (krylov)     --  (cpr);
\path (post)  --  (timeloop) coordinate[midway](null1) ;
\path (timeloop.north)+(0,0.3)  --++  (-7,0.3) -- (-7,-12.65) -| (conv2.west);
\path  (conv1.north) -- (solver.south);
\path  (res.east)--++(3,0) |-(conv1.east);
\path (conv2) -- (post);
\path (end) -- (conv2);
\end{scope}
\path [draw] (timeloop.south)--(conv1);

\end{tikzpicture}
}
\caption{Generic Algorithm Of A Reservoir Simulator.}\label{simulator}
\end{figure}
\clearpage

A crucial advantage of simulation tools over classical engineering analysis practices is the ability
to perform a large number of calculations in an efficient way. In reservoir simulation, the most taxing
component of running a simulator is usually the linear solver. Therefore, research in developing efficient
techniques in solving large system of equations is ample in the reservoir simulation literature. The main
focus is usually on developing scalable preconditioners that can be combined with iterative Krylov-based 
iterators. The industry standard preconditioner is the Constrained Pressure Residual. This preconditioner
has its roots in the works of Wallis\supercite{Wallis_1983,Wallis_1985}, the essential idea relies on decoupling
the resulting system of linear equations by isolating the elliptic dominant part of the system (pressure) from 
the hyperbolic (saturation or concentration) part of the system. 

The convergence of a Krylov iterative solver depends essentially on the preconditioner. The prototypical Krylov method, the
Conjugate Gradient can handle Symmetric-Positive-Definite matrices. For reservoir simulation cases, where matrices are ill-conditioned and
not symmetric, the Generalized Minimum Residual \textit{GMRES}, is used\supercite{roy}. 

\section{Spatial Discretization}
The discretization techniques consist an active area of research in numerical solution of PDEs. Historically reservoir simulators used 
finite-difference method to discretize the nonlinear PDE system. The current generation of reservoir simulators, both originating from
academy and industry, utilize the Two-Point Flux-Approximation (TPFA), which is based on a finite volume method. Finite-volume methods
originate from conservation laws. A more accurate method for high-resolution grids with non-orthogonal oriented 
grids with respect to the permeability tensor is the Multi-Point Flux-Approximation (MPFA) \cite{mpfa}. TPFA is not consistent for grids
that are not K-orthogonal and a more accurate discretization technique must be used. In this brief description the discussion will be
restricted to the simple case of orthogonal grids and the TPFA will be discussed. 

The formulation used in \texttt{UTCOMPRS} can be described by the following system of PDEs\supercite{phdfernandes} (see the nomenclature for symbols definition \ref{nomenclature}):
\begingroup\makeatletter\def\f@size{12}\check@mathfonts
\begin{equation}
	\begin{cases}
		\frac{1}{V_{b}} \frac{\partial N_{k}}{\partial t} = \sum_{j=2}^{n_{p}}\Big\{\nabla \cdot\Big( x_{kj}\xi_{j}\frac{k_{rj}}{\mu_{j}}\tensor{K}\cdot\nabla\Phi_{j}\Big) + \nabla \cdot \Big(\phi S_{j}\xi_{j}\tensor{\Lambda}_{kj}\cdot\nabla x_{kj}\Big)\Big\} - \frac{q_{k}}{V_{b}}, \ k=1,...,n_{c}\\\\
		\frac{1}{V_{b}} \frac{\partial N_{w}}{\partial t} = \nabla \cdot \Big(\xi_{w}\frac{k_{rw}}{\mu_{w}}\tensor{K}\cdot \nabla\Phi_{w}\Big) - \frac{q_{w}}{V_{b}}
	\end{cases}
\label{systempde}
\end{equation}\endgroup
By taking an integral of the equation \ref{systempde} over a control volume $V_{i}$ (using the water equation for simplicity), the right hand side can be converted to surface integral using \textit{Gauss's theorem}:
\begin{equation}
\int_{V_{i}}\Big[\nabla \cdot \Big(\xi_{w}\frac{k_{rw}}{\mu_{w}}\tensor{K}\cdot \nabla\Phi_{w}\Big) - \frac{q_{w}}{V_{i}}\Big] \ dV = \int_{S_{i}}\Big(\xi_{w}\frac{k_{rw}}{\mu_{w}}\tensor{K}\cdot \nabla\Phi_{w}\Big)\cdot \vec{dA} - q_{w,i}
\end{equation}
To further approximate the surface integral, a summation over the control volume faces is performed:
\begin{equation}
\int_{S_{i}}\Big(\xi_{w}\frac{k_{rw}}{\mu_{w}}\tensor{K}\cdot \nabla\Phi_{w}\Big)\cdot \vec{dA} \approx \sum_{s=1}^{n_{f}} \Big(\xi_{w}\frac{k_{rw}}{\mu_{w}}\tensor{K}\cdot \nabla\Phi_{w}\Big)\cdot\vec{A_{s}}
\label{surface}
\end{equation}

\begin{figure}[htb]
\raggedright
\resizebox{13cm}{!}{\begin{tikzpicture}
\colorlet{veccol}{green!45!black}
\colorlet{myred}{red!90!black}
\colorlet{myblue}{blue!90!black}
\colorlet{mypurple}{blue!50!red!80!black!80}
\tikzstyle{vector}=[->,very thick,veccol]
\usetikzlibrary{arrows.meta}
\tikzstyle{thin arrow}=[dashed,thin,-{Latex[length=4,width=3]}]
\definecolor{alizarin}{rgb}{0.82, 0.1, 0.26}
\definecolor{ceruleanblue}{rgb}{0.16, 0.32, 0.75}

 \path[fill=blue!30] (0,0) coordinate(p1) --  ++(1:2.5) coordinate(p2)
 -- ++(30:0) coordinate(p3) --
 ++(-100:3) coordinate(p4) --  ++(170:3) coordinate(p5);

 \coordinate (c1) at (barycentric cs:p1=1,p2=1,p3=1,p4=1,p5=1);
 \node at (c1) [xshift=-0.6cm, yshift=0.6cm] {$\Omega_{i}$};
 \draw[fill=red](c1) circle (1 pt) node  [above] {$p_{i}$};

 \foreach \X [count=\Y] in {2,...,6}
 {\ifnum\X=6
   \path (p\Y) -- (p1) coordinate[pos=0](a\Y) coordinate[pos=1](a1)
   coordinate[pos=0.5](m1);
   \draw (a\Y) -- (a1);
   %\draw[-latex] (m1) -- ($ (m1)!1.2cm!90:(p1) $) node[pos=1.2]{$a_{\Y}$};
  \else
   \path (p\Y) -- (p\X) coordinate[pos=0](a\Y) coordinate[pos=1](a\X)
   coordinate[pos=0.5](m\X);
   \draw (a\Y) -- (a\X);
   %\draw[-latex] (m\X) -- ($ (m\X)!1.2cm!90:(p\X) $) node[pos=1.2]{$a_{\Y}$};
  \fi}

 \path[fill=orange!30] (p3) coordinate(s1) --  ++(p1) coordinate(s2) -- ++(-2.5:1.5) coordinate(s3) 
 -- ++(-45:2.5) coordinate(s4) --
 (p4) coordinate(s5) ;

 \coordinate (c2) at (barycentric cs:s1=1,s2=1,s3=1,s4=1,s5=1);
 \node at (c2) [xshift=0.6cm, yshift=-0.6cm] {$\Omega_{j}$};
 \draw[fill=black](c2) circle (1 pt) node [above] {$p_{j}$};

 \foreach \X [count=\Y] in {2,...,6}
 {\ifnum\X=6
   \path (s\Y) -- (s1) coordinate[pos=0](b\Y) coordinate[pos=1](b1)
   coordinate[pos=0.5](t1);
   \draw (b\Y) -- (b1);
   %\draw[-latex] (m1) -- ($ (m1)!1.2cm!90:(p1) $) node[pos=1.2]{$a_{\Y}$};
  \else
   \path (s\Y) -- (s\X) coordinate[pos=0](b\Y) coordinate[pos=1](b\X)
   coordinate[pos=0.5](t\X);
   \draw (b\Y) -- (b\X);
   %\draw[-latex] (m\X) -- ($ (m\X)!1.2cm!90:(p\X) $) node[pos=1.2]{$a_{\Y}$};
  \fi}

  \coordinate (s1+s5) at ($0.5*(s1)+0.5*(s5)$);
  \draw[vector,thin arrow,alizarin] (barycentric cs:p1=1,p2=1,p3=1,p4=1,p5=1) -- (s1+s5) node[midway,above] {$\vec{\vb{c}}_{i,j}$};
  \draw[vector,thin arrow,ceruleanblue] (s1+s5) -- ++(0.5,0) coordinate(s1+s5) node[midway,above] {$ \ \ \vec{\vb{n}}_{i,j}$};

\node at (-3,0.5) {$\begin{aligned}
	T_{i,j} &= A_{i,j}\frac{\mathbf{\tensor{K}}_{i}\cdot\vec{c}_{i,j}\cdot\vec{n}_{i,j}}{|\vec{c}_{i,j}|^{2}}\\
		T_{ij}	&= [T_{i,j}^{-1}+T_{j,i}^{-1}]^{-1}\\
		&\sum_{j}T_{ij}(p_{i}-p_{j}) = q_{i}
		\end{aligned}$};

\end{tikzpicture}
}
\caption{Two-Point Flux-Approximation.}\label{tpfa-tikz}
\end{figure}

There are many ways to discretize the gradient operator below we consider one approximation. A typical way is to use one-sided finite-difference to express the gradient of the potential as the difference between the potential $\Pi_{i,j}$
at the face centroid between gridblock $i$ and $j$ and the potential at the center inside gridblock $i$, $\Phi_{w,i}$ (assuming pressure is constant inside a gridblock):
\begin{equation}
	\Big(\xi_{w}\frac{k_{rw}}{\mu_{w}}\tensor{K}\cdot \nabla\Phi_{w}\Big)\cdot\vec{A_{s}} \approx \xi_{w}\frac{k_{rw}}{\mu_{w}}A_{i,j}\tensor{K}_{i}\cdot\frac{(\Phi_{w,i}-\Pi_{i,j})\vec{c}_{i,j}}{|\vec{c}_{i,j}|^{2}}\cdot\vec{n}_{i,j} = \xi_{w}\frac{k_{rw}}{\mu_{w}}T_{i,j} (\Phi_{w,i}-\Pi_{i,j})
	\label{flux_i}
\end{equation}
Carrying the same discretization for neighboring gridblock $j$, we get:
\begin{equation}
	\Big(\xi_{w}\frac{k_{rw}}{\mu_{w}}\tensor{K}\cdot \nabla\Phi_{w}\Big)\cdot\vec{A_{s}} \approx \xi_{w}\frac{k_{rw}}{\mu_{w}}A_{j,i}\tensor{K}_{j}\cdot\frac{(\Phi_{w,j}-\Pi_{j,i})\vec{c}_{j,i}}{|\vec{c}_{j,i}|^{2}}\cdot\vec{n}_{j,i} = \xi_{w}\frac{k_{rw}}{\mu_{w}}T_{j,i} (\Phi_{w,j}-\Pi_{j,i})
	\label{flux_j}
\end{equation}
By making the following assumptions we can simplify the final discretization and arrive at the TPFA scheme:
\begin{enumerate}[label=\color{alizarin}\Roman*.]
	\item $\Pi_{i,j} = \Pi_{j,i}$: Continuity of face potential.
	%\item $\vec{n}_{i,j} = -\vec{n}_{j,i}$: Opposite normal area vectors.
	\item\label{potequal} (\ref{flux_i}) $=-$(\ref{flux_j}): Flux from gridblock $i$ to gridblock $j$, is minus flux from gridblock $j$ to gridblock $i$.
\end{enumerate}
Let the water flux from gridblock $i$ to gridblock $j$ be $v_{w, ij}$, writing both equations \ref{flux_i} and \ref{flux_j} using this:
\begin{align}
T_{i,j}^{-1}v_{w, ij} = \xi_{w}\frac{k_{rw}}{\mu_{w}} (\Phi_{w,i}-\Pi_{i,j})\nonumber \\
T_{j,i}^{-1}v_{w, ij} = -\xi_{w}\frac{k_{rw}}{\mu_{w}}(\Phi_{w,j}-\Pi_{j,i})
\label{fluxes}
\end{align}
Summing both equations in \ref{fluxes} and using condition \ref{potequal}:
\begin{equation}
(T_{i,j}^{-1}+T_{j,i}^{-1})v_{w, ij} = \xi_{w}\frac{k_{rw}}{\mu_{w}} (\Phi_{w,i}-\Phi_{w,j})
\end{equation}
Solving for the water flux $v_{w, ij}$:
\begin{equation}
v_{w, ij} = [T_{i,j}^{-1}+T_{j,i}^{-1}]^{-1}\xi_{w}\frac{k_{rw}}{\mu_{w}} (\Phi_{w,i}-\Phi_{w,j}) = \xi_{w}\frac{k_{rw}}{\mu_{w}}T_{ij}(\Phi_{w,i}-\Phi_{w,j})
\end{equation}
Inserting the last equation for flux in \ref{surface} and plugging it back to our main PDE \ref{systempde}, we get the TPFA scheme:
\begin{equation}
	\boxed{\frac{\partial N_{w,i}}{\partial t} = \sum_{j=1}^{nb}\lambda_{w,ij}T_{ij}(\Phi_{w,i}-\Phi_{w,j})- q_{w,i}}
\label{tpfa}
\end{equation}
The terms $T_{i,j}$ and $T_{j,i}$ are refered to as \textit{half-transmissibilities} since they are associated with a half face.
See Figure \ref{tpfa-tikz} for an illustration of this approximation scheme for a simple incompressible flow.
In carrying out the approximation in equations \ref{flux_i} and \ref{flux_j}, we have neglected the dot product between the permeability tensor $\tensor{K}$ and
the gradient potential vector $\nabla\Phi_{w}$. This was done by assuming the gradient of the potential is along the face area vector $\vec{n}_{i,j}$. This is not always correct.
Therefore, the TPFA scheme is not \textit{consistent}. Still, it is used as the standard spatial discretization technique in most industrial simulators. To avoid TPFA errors the simulation
grid axes must be aligned along the permeability tensor. Grid orientation effects can distort simulation values and give incorrect results. Grids for which TPFA is appropriate are referred to as
K-orthogonal grids and they satisfy the following condition:
\begin{equation}
	\vec{n}_{i,j}\cdot\tensor{K}\cdot\vec{n}_{i,k} = 0, \ \ \forall j\neq k,
	\label{korthogonal}
\end{equation}
where $\vec{n}_{i,j}$ and $\vec{n}_{i,k}$ denote normal vectors to faces of cell number $i$.

To complete the discretization, we need to fix the value of the mobility term ($\lambda_{w,ij}=\xi_{w}\frac{k_{rw}}{\mu_{w}}$)   using upwdind scheme.
Moreover, by assuming the permeability tensor is diagonal ($K_{xx} = K_{x}, \ K_{yy} = K_{y},\ K_{zz} = K_{z}$) and aligned with the coordinate axes of the system
the TPFA will be a valid discretization technique.

The mobility term $\lambda_{w} = \xi_{w}\frac{k_{rw}}{\mu_{w}}$, is usually evaluated on face centroid using upwind scheme (value from which flow is coming):
\begin{equation}
	\lambda_{w,ij} = 
	\begin{cases}
		\lambda_{w,i}, \ \text{if}\ T_{ij}\Delta\Phi_{j,i} \leq 0\\
		\lambda_{w,j}, \ \text{if}\ T_{ij}\Delta\Phi_{j,i} > 0
	\end{cases}
	\label{upwind}
\end{equation}

For more details about discretizing the system of PDE equations in \ref{systempde}, the reader is referred to \cite{phdfernandes}, where the full
peremeability tensor is expanded along with the dispersion tensor.

\section{Temporal Discretization}
The discretization of the time variable is of significant importance to the overall solution procedure.
Explicit and implicit methods are used to treat the time derivative. The implicit procedure produces the 
system of equations that needs to be solved, while the explicit technique results in a problem that can be
solved sequentially. The explicit method can compute the state of the system at time $n+1$ exclusively from 
variables evaluated at time level $n$. While the implicit method constructs an equation for variables at level
$n+1$ that involves both variables at level $n$ and $n+1$. Mathematically the two techniques can be represented by 
the following, if $Y(t)$ represents the state of the system at time $t$:
\begingroup\makeatletter\def\f@size{12}\check@mathfonts
\begin{equation}
	\begin{cases}
		Y(t+\Delta t) = F(Y(t)), \ \text{(Explicit scheme)}\\\\
		G(Y(t), Y(t+\Delta t)) = 0, \ \text{(Implicit scheme)}\\\\
		Y(t+\Delta t) = F(Y(t+\Delta t)) + G(Y(t)), \ \text{(Mixed scheme)}
	\end{cases}
\end{equation}\endgroup
The explicit methods are easier to implement, but their fallout is a severe restriction on timestep size, 
requiring a small timestep to be stable. Implicit methods on the other hand, will result in large systems of 
equations that will take time to solve, but they allow for the selection of large timestep sizes resulting in 
overall outperformance to the explicit methods. In the reservoir simulation literature, there exists a variety of
different types of time discretization, with special emphasis on implicit and mixed (implicit on stiff variables and explicit on others) 
discretization techniques. In \texttt{UTCOMPRS}, extensive research was conducted to implement several formulations. Each formulation treats
different variables with different degrees of implicitness. A summary of the most important formulations is given in table \ref{formulations}.
Below is an example of discretizing the water equation from the system of PDEs is given for illustration.

\begin{table}[h]
	\begin{threeparttable}
	\centering
	\resizebox{\columnwidth}{!}{%
	\begin{tabular*}{1.2\textwidth}{@{\extracolsep{\fill}} l|c|c|l}
		\toprule
		Formulation Name & Implicitness Degree\textsuperscript{$\dag$} & Iterative Method \textsuperscript{$\ddag$}& Primary Variables\\
		\midrule
		Fussel \& Fussel\supercite{fussel} & IMPEC & MVNM & $\{p, N_{g}, x_{2g}, ..., x_{n_{c}g}\}$\\
		Coats\supercite{coats} & FI & NM & $\{p_{g}, S_{o} ,S_{g}, x_{3g}, ..., x_{n_{c}g}\}$ \\
		Nghiem et al.\supercite{nghiem} & IMPEC & NM & $\{p, N_{w}, N_{h}, z_{1}, ..., z_{n_{c}}\}$\\
		Young \& Stephenson\supercite{ys} & IMPEC & NM & $\{p, N_{w}, N_{h}, L_{g}, z_{1}, ..., z_{n_{c}-1}\}$\\
		Chien et al.\supercite{chien} & FI & NM & $\{p, N_{w}, N_{1}, ..., N_{n_{c}}\}$\\
		\'Acs et al.\supercite{acs} & IMPEC & OI & $\{p, N_{w}, N_{1}, ..., N_{n_{c}}\}$\\
		Watts\supercite{watts} & IMPSAT & SOI & $\{p, S_{o}, S_{g}, N_{w}, N_{1}, ..., N_{n_{c}}\}$\\
		Quandalle\supercite{quandalle} \& Savary & IMPSAT & SOI & $\{p, S_{w}, S_{g}, x_{2o}, ..., x_{n_{c}-1o}\}$\\
		Collins et al.\supercite{collins} & AIM/FI/IMPEC & NM & $\{p, N_{w}, N_{1}, ..., N_{n_{c}}\}$\\
		Branco \& Rodriguez\supercite{branco} & IMPSAT & NM & $\{p, S_{w}, S_{g}, x_{1o}, ..., x_{n_{c}-2o}\}$\\
		Wang et al.\supercite{wang} & FI & NM & $\{p, N_{w}, N_{1}, ..., N_{n_{c}}, \ln{(K_{1})}, ..., \ln{(K_{n_{c}})}\}$\\
		Haukas et al.\supercite{haukas} & IMPSAT & SOI & $\{p, S_{w}, S_{g}, \kappa_{1}..., \kappa_{n_{c}-n_{p}}\}$\\
		Ayala\supercite{ayala} & FI & QNM & $\{p, S_{w}, z_{1}, ..., z_{n_{c}-1}\}$\\
		Fernandes et al.\supercite{batista} & FI & NM & $\{p, S_{w}, z_{1}, ..., z_{n_{c}-1}\}$\\
		\bottomrule
	\end{tabular*}
	}
	\caption{Summary of the most important formulations in compositional reservoir simulation \cite{phdfernandes}.}
	\label{formulations}
	\begin{tablenotes}
	\small
	\item [\textsuperscript{$\dag$}] FI: Fully-Implicit. IMPEC: Implicit-Pressure Explicit-Concentration. AIM: Adaptive-Implicit.\\
		\phantom{x}\hspace{0.1cm}IMPSAT: Implicit-Pressure and Saturation Explicit-Mole Fraction.
	\item [\textsuperscript{$\ddag$}] NW: Newton's Method. QNM: Quasi-Newton's Method. MVNM: Minimum Variable Newton's Method.\\
		\phantom{x}\hspace{0.1cm} OI: One-Iteration. SOI: Sequential with One-Iteration.
	\end{tablenotes}
	\end{threeparttable}
\end{table}%

\section{Nonlinear Equations Solution Method}
The flow model in Figure \ref{models}, will usually result in a system of nonlinear PDEs. The discretization of this system of PDEs will
produce a system of nonlinear algebraic equations. The standard method in reservoir simulation to solve this system of nonlinear
algebraic equations is the Newton's method. For a system of $k$ nonlinear equations described by a vector valued continuously differentiable 
function $F: \ \mathbb{R}^{k}\Rightarrow\mathbb{R}^{k}$, the Newton's method for finding zeros of the function can be described as:
\begin{align}
	J_{F}(x_{n})\delta x = -F(x_{n}) \ \text{and} \ x_{n+1} = x_{n} +\delta x
\end{align}
Where, $x_{n}$ is the $n^{th}$ iterate, $x_{n+1}$ is the $n+1$ iterate, $\delta x$ is the difference between the iterates, $J_{F}(x_{n})$ is the
jacobian matrix of the vector valued function $F$ evaluated at $x_{n}$ and $F(x_{n})$ is the vector valued function evaluated at $x_{n}$.
In the context of reservoir simulation the vector valued function is the residual $R$, based on the formulation used (see table \ref{formulations}), the variables 
of this residual function are usually the pressure and concentrations, $p$, $x_{i}$. Implementing the nonlinear solver consists of writing routines that will compute
the vector valued function $R$ for a specific formulation and its derivatives with respect to the corresponding variables of the formulation. Then constructing a time loop
that will advance the simulation to the end of simulation time. The time stepping algorithm is an important part of the nonlinear solver, it will also affect the
overall efficiency of the simulator. A large time step may require many iterations within the Newton's loop, but a small timestep is not preferred since it will advance
the simulation slowly. Therefore, a balance between the size of timesteps and nonlinear iterations number is required to achieve high performance. 

Ideally, several tests must be performed to check for the convergence of the Newton's method. These are referred to as convergence criteria for residual equations. 
Most common ones in reservoir simulation are the following:

\begin{enumerate}[label=\color{alizarin}\Roman*.]
	\item\label{mb_err} Mass or Material Balance Error $MB_{\alpha} = \Delta t \bar{B}_{\alpha} \Big(\sum_{i}\mathcal{R}_{\alpha,i}/\sum_{i}PV_{i}\Big)$.
	\item\label{cnv_err} Maximum Normalized Residual $CNV_{\alpha} = \Delta t \bar{B}_{\alpha} \max_{i}\Big|\frac{\mathcal{R}_{\alpha,i}}{PV_{i}}\Big|$.
\end{enumerate}
The first test (\ref{mb_err}) is called the \textit{Material Balance Check}, the residual is the amount of material change within a gridblock. Therefore, summing over
all gridblock residuals should result in a value of zero (accounting for sinks or sources). In the equation above, $\alpha$ refers to the phase, $\Delta t$ is the timestep size. 
$\bar{B}_{\alpha}$ is the formation-volume factor of phase $\alpha$ evaluated at average pressure. $i$ is gridblock subscript and $PV$ is the pore volume used for scaling values to 
get problem independent number. A meaningful default for $MB_{\alpha}$ is $10^{-7}$.

The second nonlinear convergence check (\ref{cnv_err}) is designed to test that residual values for individual gridblocks is reasonably small. The maximum over all gridblocks of 
the scaled residual should be typically less than $0.001$.
\clearpage

\section{Linear System Solution Method}
The Newton's method can be considered as a linearization tool that produces a large system of linear equations $Ax = b$.
This system of equations must be solved to obtain the vector of increments $x$ that guide the changes to the solution vector.
The linear solver is the computational engine of the simulator and it is almost always the most taxing part of the simulator
for all cases. The ability to make various engineering studies relies on the ability to run the simulator several times in a 
reasonable amount of time. Therefore, in the computational science community in general, and in reservoir simulation in particular 
there is a vast amount of research to develop faster and scalable linear solvers. These developments continuously take advantage of
latest software and hardware technologies. The history of this topic in numerical analysis is very rich. In reservoir simulation,
some early efforts to tailor a solver for reservoir simulators started with work on direct methods \cite{direct}.
For large systems, it became very clear that direct methods are not capable of competing with iterative methods.
One of the first iterative methods to be researched in the reservoir simulation community is the \texttt{Orthomin} method \cite{orthomin}.
The method to be used with reservoir simulators must be robust with systems that pose no-symmetry and are very ill-conditioned.
Another iterative method developed in the 80s is the \texttt{Nested Factorization} \cite{nested}. This method is still in use in the 
\texttt{Eclipse} commercial simulator. The \texttt{Nested Factorization} algorithm is mainly used as a preconditioner for the conjugate gradient
and \texttt{Orthomin} solvers. The focus of research on linear solvers is on computing speed and storage requirements. 
The latest type of iterative solvers that is currently being used in almost all reservoir simulators are the Krylov based solvers (\texttt{GMRES} being the most popular \cite{gmres}).
These Krylov based solvers are covered thoroughly in \cite{saad}. The current efforts of investigations on linear solvers in the reservoir simulation community
has been on attempts to program the most efficient techniques to run on \textit{Graphics Processing Units} \cite{appleyardgpu, gpunested}.
This report will present a preconditioner that is meant to be used with a Krylov based solver to accelerate the convergence of linear system solution.
