\chapter{Implementation In UTCOMP}
The preconditioner module developed in the compositional simulator \texttt{UTCOMP}, follows a
decoupling technique simliar to the technique in the \texttt{INTERSECT IX} the next \textit{SLB}
generation of reservoir simulators. The details of the CPR implementation in \texttt{INTERSECT IX}
are presented in \cite{ix-cpr}. The CPR preconditioner in \texttt{UTCOMP} is developed as a customized
\texttt{PETSC} preconditioner, since \texttt{UTCOMP} is using \texttt{PETSC} Krylov-based iterators 
for the linear solution of resulting system of equations. During the \texttt{FGMRES} cycle of \texttt{PETSC}, 
the customized preconditioner is called and the decoupling of the pressure equation is performed. \texttt{PETSC}
provides an interface to the \texttt{hypre} library of fast preconditioners and AMG solvers\supercite{hypre}.
The AMG solver \texttt{BoomerAMG}\supercite{boomeramg}, is used to solve the pressure subsystem. The second stage
of the preconditioner consists of combining the pressure solution from AMG with the secondary variables to apply
a global \texttt{ILU(0)} preconditioning on the system as a whole. 

The decoupling is performed by looping over the main diagonal sub-blocks of the jacobian matrix 
and inverting them to build the decoupling operator. In \texttt{UTCOMP}, the inversion
of the block entries along the diagonal is performed using a direct solver, LU factorization, 
from the \texttt{LAPACK} library. 

%% These three lines  !!!
\SetNlSty{textbf}{\color{black}}{}
\newcommand*{\mycommentfont}[1]{\textcolor{black}{\ttfamily#1}}
\SetCommentSty{mycommentfont}

\section{CPR Algorithm}
\subsection{Eclipse Implementation}
%{\raggedright
%\begin{minipage}{.7\linewidth}
\begin{algorithm}[H]
    \caption{Eclipse Implementation Of CPR \cite{ecl_tech}.}
    \label{cpr_ecl}
    \DontPrintSemicolon
    \KwIn{Jacobian matrix: $\mathbf{A}$ and Nonlinear Residual: $\mathbf{R}$.}
    \KwOut{CPR preconditioned vector: $\mathbf{x}$, for the \texttt{FGMRES}.} \SetKwProg{Fn}{Function}{ is}{end}
	\While{\texttt{FGMRES \textcolor{red}{is not convergent}}}{
	\Fn{\texttt{CPR\_PRECOND}\texttt{(A, R)}} {
        \begingroup
        \color{red}
	    \textcolor{blue}{\ttfamily{\slash/ First Stage Preconditioner}}\\
	    $\mathbf{r_{p} = C^{T}GR}$ \hspace{1cm}\tcp{Restric the residual to the prssure system.}
	    $\mathbf{\mathbf{A_{p} x_{p}} = \mathbf{r_{p}}}$ \hspace{0cm}\tcp{Solve using \texttt{GMRES} with Nested-Factorization.}
	    \textcolor{black}{\texttt{call GMRES}} \hspace{2cm}$\textcolor{black}{\longleftarrow}$ \textcolor{NavyBlue}{\texttt{inner GMRES}}
            \hfill\\
	    \textcolor{blue}{\ttfamily{\slash/ Second Stage Preconditioner}}\\
	    $\mathbf{\tilde{r} = r - \mathbf{A}C\mathbf{x_{p}}}$ \hspace{0.35cm}\tcp{Expand residual by padding the other variables.}
	    $\mathbf{\tilde{x} = \tilde{M^{-1}} \tilde{r}}$ \hspace{0.9cm}\tcp{Preform Nested-Factorization on the expanded}\hspace{2.7cm}\tcp{residual.}
            \hfill\\
	    \textcolor{blue}{\ttfamily{\slash/ Combine Results}}\\
	    $\mathbf{\mathbf{x = \tilde{x} + C x_{p}}}$ \hspace{0.25cm}\tcp{Combine the results of the two stages.}
        \endgroup            
    }
    \texttt{call }\texttt{FGMRES(x)}\hspace{2cm}$\longleftarrow$ \textcolor{NavyBlue}{\texttt{outer GMRES}}
}
\end{algorithm}
%\end{minipage}
%\par
%}

\subsection{Intersect IX Implementation}
The linear solver in \texttt{Intersect IX} uses \textit{Flexible Generalized Minimum Residual} as a Krylov subspace solver, specifically a Restarted FGMRES.  
The default preconditioner in \texttt{Intersect IX} is the Constrained Pressure Residual. The two-stage preconditioner implementation in \texttt{Intersect IX}
is detailed in \cite{ix-cpr,ix-tech}. The CPR preconditioning matrix can be written for a general decoupling matrix $W$ as:
\begin{equation}
	M_{CPR}^{-1} = M^{-1}[I - AC(W^{T}AC)^{-1}W^{T}] + C(W^{T}AC)^{-1}W^{T}\approx A^{-1}
\end{equation}
The $C$ matrix is a linear operator ($\mathbb{R}^{n_{cell}}\Rightarrow\mathbb{R}^{n_{eqn}\cdot n_{cell}}$) referred to as a restriction operator.
The operator is formally written as:
\begin{equation}
	C = \begin{bmatrix}
		\mathbf{e_{p}} & & & & \\
		      & \mathbf{e_{p}} & & &\\ 
		      &  & \mathbf{e_{p}}& &\\ 
		      &  & & \mathbf{e_{p}}&\\ 
		      &  & & &\mathbf{e_{p}}\\ 
	\end{bmatrix}, \ 
	\mathbf{e_{p}} = \begin{bmatrix}
		0\\
		\vdots\\
		0\\
		1
	\end{bmatrix}
	\label{cop}
\end{equation}
Where the vector $\mathbf{e_{p}}$ is a unit vector of size $n_{eqn}$.
The matrix $M$ represents a preconditoiner for the second-stage on the whole system combined. Usually \texttt{ILU(0)} is used.
Both in \texttt{UTCOMPRS} and \texttt{Intersect IX} the decoupling operator used is $W^{T} = C^{T}DIAG^{-1}(A)$. Where $DIAG(A)$,
is the diagonal blocks of the jacobian matrix, that is zeroing every other block not along the diagonal. The pressure matrix can be written
using these operators as a matrix with size $n_{cell}\times n_{cell}$:
\begin{equation}
	A_{p} = W^{T}AC
\end{equation}
The right hand side of the pressure subsystem is produced by application of the same linear operator $M_{CPR}^{-1} \ (\mathbb{R}^{n_{eqn}n_{cell}}\Rightarrow\mathbb{R}^{n_{cell}})$ 
on the system original residual $r$.

\begin{algorithm}[H]
    \caption{Intersect Implementation Of CPR \cite{ix-tech}.}
    \label{cpr_ecl}
    \DontPrintSemicolon
    \KwIn{Jacobian matrix: $\mathbf{A}$ and Nonlinear Residual: $\mathbf{R}$.}
    \KwOut{CPR preconditioned vector: $\mathbf{x}$, for the \texttt{FGMRES}.}
    \SetKwProg{Fn}{Function}{ is}{end}
	\While{\texttt{FGMRES \textcolor{red}{is not convergent}}}{
	\Fn{\texttt{CPR\_PRECOND}\texttt{(A, R)}} {
        \begingroup
        \color{red}
	    \textcolor{blue}{\ttfamily{\slash/ First Stage Preconditioner}}\\
	    $\mathbf{r_{p} = W^{T}R}$ \hspace{1cm}\tcp{Restric the residual to the prssure system.}
	    $\mathbf{\mathbf{A_{p} x_{p}} = \mathbf{(W^{T}AC)x_{p}}=\mathbf{r_{p}}}$ \hspace{0cm}\tcp{Solve using \texttt{GMRES} with Nested-Factorization.}
	    \textcolor{black}{\texttt{call GMRES}} \hspace{2cm}$\textcolor{black}{\longleftarrow}$ \textcolor{NavyBlue}{\texttt{inner GMRES}}
            \hfill\\
	    \textcolor{blue}{\ttfamily{\slash/ Second Stage Preconditioner}}\\
	    $\mathbf{\tilde{r} = r - \mathbf{A}C\mathbf{x_{p}}}$ \hspace{0.35cm}\tcp{Expand residual by padding the other variables.}
	    $\mathbf{\tilde{x} = M^{-1} \tilde{r}}$ \hspace{0.9cm}\tcp{Preform ILU(0) Factorization on the expanded}\hspace{2.7cm}\tcp{residual.}
            \hfill\\
	    \textcolor{blue}{\ttfamily{\slash/ Combine Results}}\\
	    $\mathbf{\mathbf{x = \tilde{x} + C x_{p}}}$ \hspace{0.25cm}\tcp{Combine the results of the two stages.}
        \endgroup            
    }
    \texttt{call }\texttt{FGMRES(x)}\hspace{2cm}$\longleftarrow$ \textcolor{NavyBlue}{\texttt{outer GMRES}}
}
\end{algorithm}

